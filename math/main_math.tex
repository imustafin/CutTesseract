\documentclass[12pt, a4paper, twoside]{report}
\usepackage[utf8]{inputenc}
\usepackage[T1,T2A]{fontenc}

\usepackage[russian, english]{babel}
\usepackage{xcolor} %for transparency
\usepackage{graphicx}

\usepackage{indentfirst}
\usepackage{amsmath, amsthm, amssymb} %theorem package

\usepackage{microtype}

\newcommand{\HRule}{\rule{\linewidth}{0.5mm}}
%\setcounter{section}{1}
\renewcommand{\thesection}{\arabic{section}}

%Rename contents - magic here
\addto\captionsenglish{% Replace "english" with the language you use
	  \renewcommand{\contentsname}%
	      {Оглавление}%
      }
%END rename contents

\newtheorem{lem}{Утверждение} %THEOREMS HERE
\usepackage[top=2.5cm, bottom=2.5cm, right=3.5cm]{geometry}

\begin{document}

\begin{titlepage}
\begin{center}
\fontencoding{T1}\selectfont

\textsc{\LARGE Liceum Boarding School number 2}\\[1.5cm]

\textsc{\Large Some crappy project}\\[0.5cm]

% Title
\HRule \\[0.4cm]
{ \huge \bfseries Tesseract and hyperplane investigation\\[0.4cm] }

\HRule \\[1.5cm]

% Author and supervisor
\noindent
\begin{minipage}{0.4\textwidth}
\begin{flushleft} \large
\emph{Authors:}\\
Ilgic \textsc{Mustafin} \newline
\fontencoding{T2A}\selectfont
Гриша \textsc{Maxxx} \newline %USE RUSSIAN FONT
\fontencoding{T1}\selectfont %USE LATIN HERE
Mirolin \textsc{M} 
and \newline
Arthur \textsc{Nugmanov} 
\end{flushleft}
\end{minipage}%
\begin{minipage}{0.4\textwidth}
\begin{flushright} \large
\emph{Supervisor:} \\
Marsel \textsc{Abiy}
\end{flushright}
\end{minipage}

\vfill

% Bottom of the page
{\large \today}

\end{center}
\end{titlepage}
 %Load titlepage so we do not mess up the numeration

\tableofcontents

\newpage
\section{Введение}
Введение, тезисы, примерное описание работы.
Данная работа ставит своей целью создать математическую модель пересечения четырехмерного гиперкуба или тессеракта трехмерной гиперплоскостью, а также создать программную имплементацию данной математической задачи. Предполагается рассмотреть варианты возможных сечений данного гиперкуба.

Как известно, гиперкуб представляет собой куб в четырехмерном пространстве, может также называться тетракубом и тессерактом. В данной работе будет рассматриваться представление данной фигуры в виде множества точек в Евклидовом пространстве $(x_1,x_2,x_3,x_4) \in R^4$, таких что $ x_i\in [ -1,1 ].$ Тессеракт ограничивают 8 гиперплоскостей, т.е. подпространств, на единицу меньшим, чем объемлющее пространство.

\begin{figure}[h!]
	\center
	\framebox{\includegraphics[scale=0.5]{./tesseract_fig1.png}}
	\clearpage
\end{figure}
\section{Тессеракт, или четырехмерный гиперкуб}
Конечно, невозможно представить графически, как будет выглядеть данный гиперкуб в четырехмерном пространстве, не выходя из трехмерного. Можно лишь представить проекцию данного тессеракта на трехмерную плоскость. В случае проекции невозможно определить точки сечения, поэтому все действия нами выполнялись в четырехмерных координатах для получения трехмерного сечения, представляющего из себя многогранник.

Далее будет рассмотрен довольно простой способ представления проекции тессеракта на трехмерную плоскость. Этот метод позволяет представить примерно, как выглядит рабочий куб, но, как уже отмечалось выше, данный метод не позволяет определить точки пересечения гиперплоскости с четырехмерным кубом.
\subsection{Построение проекции гиперкуба на плоскость}
Попытаемся представить себе, как будет выглядеть гиперкуб, не выходя из трёхмерного пространства. Возьмем одномерное пространство (линию), выделим на нем отрезок длиной $a$. Теперь перенесем его на двумерную плоскость и на расстоянии $a$ от него нарисуем параллельный ему отрезок той же длины, затем соединим концы. Получится квадрат. Повторив эту операцию с двумерной плоскостью, получим трехмерный куб, а если применить ту же операцию к трехмерному пространству, то получится гиперкуб, т.е. его можно представить, как два трехмерных куба, лежащих в параллельных трехмерных пространствах, с попарно соединенными вершинами.

\begin{figure}[h!]
	\framebox{\includegraphics[scale=0.6]{./make_tess.png}}
	\clearpage
\end{figure}

Как видим, отрезок $AB$ в результате параллельного переноса превращается в квадрат, т.е. уже куб 2 измерения. По аналогии куб $ABCDEFGH$ образует проекцию гиперкуба на трехмерную плоскость.

Если задуматься о том, как будет выглядеть сечение тессеракта в четырехмерном пространстве, можно прийти к следующей аналогии: сечением одномерного объекта будет точка, двумерного -  отрезок, трехмерного - плоскость. По аналогии сечением четырехмерного объекта является трехмерный объект. Наша задача состоит как раз в поиске принципа построения данных сечений.

\section{Мат. аппарат}
Перейдем к описанию математической модели построения данных сечений.

Для начала нам нужно задать уравнение гиперплоскости сечения. Для этого на нужно либо 4 четырехмерные точки, либо четырехмерный вектор и точка. Таким образом, мы зададим плоскость сечения. После получения уравнения гиперплоскости сечения мы проводим проверку каждого ребра куба на пересечение с гиперплоскостью сечения. Далее мы проверяем, если точки сечения лежат на одной грани, то мы их соединяем (создаем из них сегмент), после чего идет отображение полученных сегментов по двум точкам и получение финального сечения. Однако, оно все еще в четырехмерных координатах, поэтому мы берем нормаль гиперплоскости сечения и вместе с сечением поворачиваем так, чтобы она стала параллельна четвертой оси координат (0, 0, 0, 1). Для достижения данной задачи нами используются матрицы поворота, о которых будет сказано в дальнейшем. В результате этих манипуляций четвертая координата у всех точек сечения становится одинаковой, т.е. мы можем её не учитывать при выводе полученных граней сечения на экран. На данный момент сечение представляет собой трехмерное геометрическое тело, вполне доступное для наблюдения и понимания.

Разберем подробнее процесс задания уравнения гиперплоскости сечения.
\subsection{Уравнение гиперплоскости сечения}
Как было сказано, мы используем два метода для решения данной задачи. Разберем, как задается уравнение гиперплоскости по четырем 4D точкам.
Известно, что уравнение искомой плоскости записывается в виде:
$$ax + by + cz + dw + e=0,$$ где $(a,b,c,d,e)$ - коэффициенты, а $(x,y,z,w)$ - координаты возможных точек. Точка принадлежит гиперплоскости, если выполняется равенство.

Требуется по данным точкам $$A(x_1,y_1,z_1,w_1), B(x_2,y_2,z_2,w_2), C(x_3,y_3,z_3,w_3), D(x_4,y_4,z_4,w_4)$$ составить уравнение гиперплоскости, т.е. найти нужные коэффициенты.

Уравнением плоскости будет являться: $$ \left|
		\begin{array}{cccc}
			x-x_1 & y-y_1 & z-z_1 & w-w_1     \\
			x_2-x_1 & y_2-y_1 & z_2-z_1 & w_2-w_1    \\
			x_3-x_1 & y_3-y_1 & z_3-z_1 & w_3-w_1      \\
			x_4-x_1 & y_4-y_1 & z_4-z_1 & w_4-w_1 
		\end{array}
	\right|=0

$$

Далее разберем другой способ. Пусть даны точка $A(x_1,y_1,z_1,w_1)$, и вектор $\overrightarrow N(x_2,y_2,z_2,w_2)$. Объявим гиперплоскость $P(ax + by + cz + dw + e=0)$.
Вектор $\overrightarrow N$ будет являться нормалью к данной плоскости, т.е. будет справедливо следующее равенство: $$a=x_2, b=y_2, c=z_2, d=w_2.$$ Останется найти коэффициент $e$, который будет находиться подстановкой координат данной точки $A$ в уравнение гиперплоскости. Т.е. $$e=-(x_2x_1+y_2y_1+z_2z_1+w_2w_1)$$
Таким образом, мы объявили гиперплоскость, используя введенные данные.
\subsection{Нахождение пересечения данной гиперплоскости с ребрами гиперкуба}
Для нахождения точек сечения надо найти пересечения данной гиперплоскости $P$ и всех ребер куба.
Допустим, имеем гиперкуб с координатами $$(\pm1, \pm1, \pm1, \pm1),$$ 16 вершин или 32 ребра. Для начала проверим, являются ли вершины гиперкуба точками сечения. Для этого берем последовательно все 16 вершин гиперкуба и подставляем их координаты в уравнение гиперплоскости.
\newline
$\begin{equation*}
	A:=(x_1,y_1,z_1,w_1) \\
P:=(ax + by + cz + dw + e = 0) \\
\left[
	\begin{array}{ccc}
		ax_1 + by_1 + cz_1 + dw_1 + e = 0 & \mbox{Вершина гиперкуба принадлежит сечению} 
		\\
		ax_1 + by_1 + cz_1 + dw_1 + e > 0 & \mbox{Вершина гиперкуба лежит "выше" плоскости сечения}
		\\
		ax_1 + by_1 + cz_1 + dw_1 + e < 0 & \mbox{Вершина гиперкуба лежит "ниже" плоскости сечения} 
	\end{array}
\end{equation*}$
\\

Теперь, если данные точки, принадлежащие одному ребру, лежат по разные стороны от сечения, то между ними есть точка, принадлежащая сечению. Если обе точки лежат по одну сторону от сечения, то значит, данное ребро и плоскость сечения не имеют общих точек. Возможна также ситуация, когда вершина гиперкуба сама является точкой сечения.
Обратимся к случаю, когда данные точки лежат по разные стороны от сечения. Требуется найти точку пересечения секущей гиперплоскости и данного ребра. Пусть мы работаем с ребром $AB$ гиперкуба и плоскостью сечения $P(ax + by + cz + dw + e = 0)$, для которой мы точно знаем, что она имеет общую точку с ребром.
\\
$\begin{equation}
A=(x_1,y_1,z_1,w_1)\\ 
B=(x_2,y_2,z_2,w_2) \\
\overrightarrow{AB}=((x_2-x_1),(y_2-y_1),(z_2-z_1),(w_2-w_1))
\end{equation}$

\\
Зададим параметрическое уравнение прямой $AB$ через точку $A$ и вектор $\overrightarrow{AB}$.
\\
$
\begin{equation}
	A(x_1,y_1,z_1,w_1) \\
	\overrightarrow{AB}=(a_x,a_y,a_z,a_w) \\
	\left\{
		\begin{array}{ccc}
			x=x_1+a_xt \\
			y=y_1+a_yt \\
			z=z_1+a_zt \\
			w=w_1+a_wt
		\end{array}
\end{equation}
$

\
Требуется найти неизвестный коэффициент $t$. Находим его следующим образом, используя уравнение гиперплоскости:
\\
$$	P(ax+by+cz+dw+e=0)$$ 
$$	t=-\frac{(ax_1+by_1+cz_1+dw_1+e)}{(aa_x+ba_yca_z+da_w)}$$

Теперь параметрическое уравнение задано, требуется найти координаты точки пересечения данного ребра с полученной прямой. Для нахождения координат точек пересечения надо вычислить значения $(x,y,z,w)$, входящие в состав системы параметрического уравнения прямой. Найденная точка лежит между вершинами данного ребра тессеракта и принадлежит гиперплоскости сечения. 

Требуется повторить данную операцию для всех 32 ребер, которые имеют общие точки с данной плоскостью сечения.
\subsection{Построение геометрического тела сечения, поворот}
После нахождения всех точек, требуется соединить их ребрами. Для этого перебираем все возможные пары точек из полученного множества. Для каждой пары применим следующие критерии нахождения их на одном ребре - чтобы две данные точки лежали на одном ребре, необходимо, чтобы хотя бы две их координаты были одинаковы и равны соответственно $1$ и $-1$.
Если такое условие выполняется, то две полученные точки можно соединить ребром. Множество полученных ребер и будет ограничивать все множество точек полученного сечения.

Следующим шагом требуется повернуть плоскость сечения таким образом, чтобы нормаль к данной плоскости стала параллельной четвертой координатной оси $Ow$. Для осуществления данной цели используются матрицы поворота точки параллельно плоскости. Составляя нужную композицию поворотов, можно достичь поставленной выше задачи.

Пусть дана система координат $Oxyzw$. Определим матрицы поворота относительно плоскостей:
\\

$
M_{xy}(\alpha)=
\begin{equation}
	\left(
		\begin{array}{cccc}
		\cos \alpha & -\sin \alpha & 0 & 0 \\
		\sin \alpha & \cos \alpha & 0 & 0 \\
		0 & 0 & 1 & 0 \\
		0 & 0 & 0 & 1
	\end{array}\right)
\end{equation}
$
\\ 

$
M_{zw}(\alpha)=
\begin{equation}
	\left(
		\begin{array}{cccc}
		1 & 0 & 0 & 0 \\
		0 & 1 & 0 & 0 \\
		0 & 0 & \cos \alpha & -\sin \alpha \\
		0 & 0 & \sin \alpha & \cos \alpha
	\end{array}\right)
\end{equation}
$
\\

$
M_{yz}(\alpha)=
\begin{equation}
	\left(
		\begin{array}{cccc}
		1 & 0 & 0 & 0 \\
		0 & \cos \alpha & -\sin \alpha & 0 \\
		0 & \sin \alpha & \cos \alpha & 0 \\
		0 & 0 & 0 & 1
	\end{array}\right)
\end{equation}
$
\\

Определим произвольную матрицу поворота $M(\phi)$ как композицию поворотов $$M(\phi)=M_{xy}(\alpha)\circ M_{zw}(\beta)\circ M_{yz}(\gamma).$$
Таким образом, чтобы повернуть нормаль плоскости сечения параллельно $Ow$, нам потребуется повернуть последовательно в плоскостях $xy, xz, yz$.

После данных манипуляций получаем трехмерное результирующее сечение, которое и выводится на экран. Данное сечение не имеет координаты $w$, и представляет собой геометрическое тело в трехмерном Евклидовом пространстве.
Данное сечение имеет одинаковую координату $w$, и, как следствие, идентично со своей проекцией на трехмерное пространство.
\newpage
\section{Некоторые свойства сечений}
Определим некоторые свойства, являющиеся общими для всех полученных сечений тессеракта гиперплоскостью. Докажем следующее утверждение:
\begin{lem}[О гранях сечения]
Грани сечения, полученные пересечением противоположных ячеек тессеракта секущей плоскостью, параллельны.	
\end{lem}
\begin{proof}[Доказательство] 
	Определим две плоскости $P_1$ и $P_2$ как пересечение двух гиперплоскостей в четырехмерном пространстве: \\

$
\begin{equation}
	P_1 \left\{
		\begin{array}{lc}
			ax+by+cz+dw+e=0 \\
			w=1
		\end{array}
\end{equation}
$
\\

$
\begin{equation}
	P_2 \left\{
		\begin{array}{lc}
			ax+by+cz+dw+e=0 \\
			w=-1
		\end{array}
	\end{equation}
$

Как видно из приведенных выше уравнений, мы рассматриваем противоположные ячейки гиперкуба. 
Заметим, что для существования ячеек гиперкуба для них должно выполняться утверждение $a^2+b^2+c^2\ne0$ 
Не умаляя общности, примем, что $a\ne 0$ \\
Теперь, определим произвольную точку $M_1(x_1;y_1;z_1;1)$, такую, что $M_1\in P_1$. Тогда верно следующее равенство:
$$
ax_1+by_1+cz_1+d+e=0
$$
Найдем формулу для вычисления следующей произвольной точки $M_2$, такой, что $M_2 \in P_1$. Пусть: \\
\begin{align*}
	&ax+by=0 \\ 
&ax=-by \\
&x=-bk \\
&y=ak 
\end{align*}
Тогда $M_2(x_1-bk;y_1+ak;z_1;1) \in P_1.$ Проделаем то же самое для определения третьей точки $M_3$. \\
\begin{align*}
	&ax+cz=0 \\
 &x=-cn \\
 &z=an
\end{align*}
Тогда $M_3(x_1-cn;y_1;z_1+an;1) \in P_1$. 
\newpage
Теперь зададим вектора:\\ 

\begin{align*}
	&\overrightarrow{M_1M_2}=(-bk;ak;0;0) \\
	&\overrightarrow{M_1M_3}=(-cn;0;an;0) 
\end{align*}
Далее проделаем то же самое для плоскости $P_2$. Таким же образом определим на ней три точки, через которые проведем также два вектора.
\begin{align*}
	&M_4(x_2;y_2;z_2;-1) \\
 &M_4 \in P_2 \\
	&ax_2+by_2+cz_2+d+e=0 \\
	&a \ne0 \\
	&ax+by=0 \\
	&x=-bk \\
	&y=ak 
\end{align*}
Определим точку $M_5(x_2-bk;y_2+ak;z_2;-1) \in P_2$. 
\begin{align*}
	&ax+cz=0 \\
	&x=-cn \\
	&z=an
\end{align*}
Определим точку $M_6(x_2-cn;y_2;z_2+an;-1) \in P_2$. И, наконец, определим векторы $\overrightarrow{M_4M_5}$ и $\overrightarrow{M_4M_6}$.
\begin{align*}
	&\overrightarrow{M_4M_5}=(-bk;ak;0;0) \\
	&\overrightarrow{M_4M_6}=(-cn;0;an;0) 
\end{align*}
Теперь рассмотрим векторы, принадлежащие плоскостям $P_1$ и $P_2$: $\overrightarrow{M_1M_2}, \overrightarrow{M_1M_3}$ и $\overrightarrow{M_4M_5}, \overrightarrow{M_4M_6}$.
\begin{align*}
	\overrightarrow{M_1M_2}&=(-bk;ak;0;0) \\
	\overrightarrow{M_1M_3}&=(-cn;0;an;0) \\
	\overrightarrow{M_4M_5}&=(-bk;ak;0;0) \\
	\overrightarrow{M_4M_6}&=(-cn;0;an;0)
\end{align*}
Как видим, пары неколлинеарных векторов в двух плоскостях равны, а значит и плоскости, задаваемые ими, параллельны. Данное свойство параллельности не претерпевает изменений при повороте двух рассматриваемых плоскостей. Таким образом, сечение тессеракта гиперплоскостью имеет хотя бы одну пару параллельных прямых.

\end{proof}

\section{Примеры сечений}
\subsection{Октаэдр}
\framebox{\includegraphics[scale=0.3]{./a4.png}} 
\\
Геометрическое тело октаэдр.
Правильный октаэдр - восьмигранник, состоящий из восьми треугольников. Сечение получается из точки $A(0,0,0,0)$ и вектора $\overrightarrow a=(1,1,1,1)$
\subsection{Шестиугольная призма}
\framebox{\includegraphics[scale=0.3]{./6prism1.png}}
\\

Получается из следующих точек:
$
\begin{array}{cccc}
	0 & 1 & 0 & 1 \\
	1 & 0 & 1 & 1 \\
	0 & 1 & 1 & 1 \\
	0 & 0 & 0 & 0
\end{array}
$
\subsection{Прочие сечения}

\framebox{\includegraphics[scale=0.5]{./cuttt1.png}}
\hline

\framebox{\includegraphics[scale=0.5]{./cuttt2_tess.png}}

Это же сечение, но без описанного вокруг него отображения тессеракта:

\framebox{\includegraphics[scale=0.3]{./cuttt2_notess.png}}

\section{Заключение}
Таким образом, наше исследование ставило своей целью исследовать возможности построения сечений четырехмерного гиперкуба трехмерной гиперплоскостью. Данная задача была нами достигнута. Была составлена математическая модель и на ее основе написана программа. Сечения гиперкуба строятся в четырехмерных координатах, и, как следствие, являются трехмерными геометрическими телами.
\section{Список используемой литературы}
\begin{thebibliorraphy}
https://ru.wikipedia.org/wiki/%D0%A2%D0%B5%D1%81%D1%81%D0%B5%D1%80%D0%B0%D0%BA%D1%82 
\\
http://www.coolreferat.com/%D0%9A%D0%BE%D0%BC%D0%BF%D1%8C%D1%8E%D1%82%D0%B5%D1%80%D0%BD%D0%BE%D0%B5_%D0%BF%D1%80%D0%B8%D0%BB%D0%BE%D0%B6%D0%B5%D0%BD%D0%B8%D0%B5_%D0%A1%D0%B5%D1%87%D0%B5%D0%BD%D0%B8%D0%B5_%D0%BC%D0%BD%D0%BE%D0%B3%D0%BE%D0%B3%D1%80%D0%B0%D0%BD%D0%BD%D0%B8%D0%BA%D0%BE%D0%B2_%D1%87%D0%B0%D1%81%D1%82%D1%8C=3
\section{Приложения}
Программный код. Компилировать на Java 7 \\
Программу можно скачать с созданного репозитория по адресу: \\
http://imustafin.github.io/CutTesseract/ \\
\end{document}

