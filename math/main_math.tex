\documentclass[12pt, a4paper, twoside]{report}
\usepackage[utf8]{inputenc}
\usepackage[T1,T2A]{fontenc}

\usepackage[russian, english]{babel}
\usepackage{xcolor} %for transparency
\usepackage{graphicx}

\usepackage{indentfirst}

\newcommand{\HRule}{\rule{\linewidth}{0.5mm}}
%\setcounter{section}{1}
\renewcommand{\thesection}{\arabic{section}}

%Rename contents - magic here
\addto\captionsenglish{% Replace "english" with the language you use
	  \renewcommand{\contentsname}%
	      {Оглавление}%
      }
%END rename contents

\begin{document}

\begin{titlepage}
\begin{center}
\fontencoding{T1}\selectfont

\textsc{\LARGE Liceum Boarding School number 2}\\[1.5cm]

\textsc{\Large Some crappy project}\\[0.5cm]

% Title
\HRule \\[0.4cm]
{ \huge \bfseries Tesseract and hyperplane investigation\\[0.4cm] }

\HRule \\[1.5cm]

% Author and supervisor
\noindent
\begin{minipage}{0.4\textwidth}
\begin{flushleft} \large
\emph{Authors:}\\
Ilgic \textsc{Mustafin} \newline
\fontencoding{T2A}\selectfont
Гриша \textsc{Maxxx} \newline %USE RUSSIAN FONT
\fontencoding{T1}\selectfont %USE LATIN HERE
Mirolin \textsc{M} 
and \newline
Arthur \textsc{Nugmanov} 
\end{flushleft}
\end{minipage}%
\begin{minipage}{0.4\textwidth}
\begin{flushright} \large
\emph{Supervisor:} \\
Marsel \textsc{Abiy}
\end{flushright}
\end{minipage}

\vfill

% Bottom of the page
{\large \today}

\end{center}
\end{titlepage}
 %Load titlepage so we do not mess up the numeration

\tableofcontents

\newpage
\section{Введение}
Введение, тезисы, примерное описание работы.
Данная работа ставит своей целью создать математическую модель пересечения четырехмерного гиперкуба или тессеракта трехмерной гиперплоскость, а также создать программную имплементацию данной математической задачи. Предполагается рассмотреть варианты возможных сечений данного гиперкуба.

Как известно, гиперкуб представляет собой куб в четырехмером пространстве, может также называться тетракубом и тессерактом. В даннной работе будет рассматриваться представление данной фигуры в виде множества точек в Евклидовом пространстве $(\pm 1,\pm 1,\pm 1, \pm 1)$. Тессеракт ограничивают 8 гиперплоскостей, т.е.подпространств, на единицу меньшей чем объемлющее пространство.

\begin{figure}[h!]
	\center
	\framebox{\includegraphics[scale=0.5]{./tesseract_fig1.png}}
	\clearpage
\end{figure}
\section{Тессеракт, или четырехмерный гиперкуб}
Конечно, невозможно представить графически, как будет выглядеть данный гиперкуб в четырехмерном пространстве, не выходя из трехмерного. Можно лишь представить проекцию данного тессеракта на трехмерныю плоскость. В случае проекции невозможно определить точки сечения, поэтому все действия нами выполнялись в четырехмерных координатах для получения трехмерного сечения, представляющего из себя многогранник.

Далее будет рассмотрен довольно простой способ представления проекции тессеракта на трехмерную плоскость. Этот метод позволяет представить примерно, как выглядит рабочий куб, но, как уже отмечалось выше, данный метод не позволяет определить точки пересечения гиперплоскости с четырехмерным кубом.
\subsection{Построение проекции гиперкуба на плоскость}
Попытаемся представить себе, как будет выглядеть гиперкуб, не выходя из трёхмерного пространства. Возьмем одномерное пространство (линию), выделим на нем отрезок длиной $а$. Теперь перенесем его на двумерную плоскость и на расстоянии $а$ от него нарисуем параллельный ему отрезок той же длины, затем соединим концы. Получится квадрат. Повторив эту операцию с двумерной плоскостью, получим трехмерный куб, а если применить ту же операцию к трехмерному пространству, то получится гиперкуб т.е. его можно представить, как два трехменых куба, лежащих в парралельных трехмерных пространствах, с попарно соединенными вершинами. 

\begin{figure}[h!]
	\framebox{\includegraphics[scale=0.6]{./make_tess.png}}
	\clearpage
\end{figure}

Как видим, отрезок $AB$ в результате параллельного переноса превращается в квадрат, т.е. уже куб 2 измерения. По аналогии куб $ABCDEFGH$ образует проекцию гиперкуба на трехмерную плоскость.

Если задуматься о том, как будет выглядеть сечение тессеракта в четырехмерном пространстве, можно прийти к следующей аналогии: сечением одномерного объекта будет точка, двумерного -  отрезок, трехмерного - плоскость. По аналогии сечением четырехмерного объекта является трехмерный объект. Наша задача состоит как раз в отыскании принципа построения данных сечений.

\section{Мат. аппарат}
Перейдем к описанию математической модели построения данных сечений.

Для начала нам нужно задать уравнение гиперплоскости сечения. Для этого на нужно либо 4 четырехмерные точки, либо четырехмерный вектор и точку. Таким образом мы зададим плоскость сечения. После получения уравнения гиперплоскости сечения мы проводим проверку каждого ребра куба на пересечение с гиперплоскостью сечения. Далее мы проверяем, если точки сечения лежат на одной грани, то мы их соединяем (создаем из них сегмент), после чего идет отрисовка полученных сегментов по двум точкам и получение финального сечения. Однако, оно все еще в четырехмерных координатах, поэтому мы берем нормаль гиперплоскости сечения и вместе с сечением поворачиваем так, чтобы она стала параллельна четвертой оси координат (0, 0, 0, 1). Для достижения данной задачи нами используются матрицы поворота, о которых будет сказано в дальнейшем. В результате этих манипуляций четвертая координата у всех точек сечения становится однаковой, т.е. мы можем её не учитывать при выводе полученных граней сечения на экран. На данный момент сечение представляет из себя трехмерное геометрическое тело, вполне доступное для наблюдения и понимания.

Разберем подробнее процесс задания уравнения гиперплоскости сечения.
\subsection{Уравнение гиперплоскости сечения}
Как было сказано, мы используем два метода для решения данной задачи. Разберем, как задается уравнение гиперплоскости по четырем 4D точкам.
Известно, что уравнение искомой плоскости записывается в виде:
$$ax + by + cz + dw + e=0,$$ где $(a,b,c,d,e)$ - коэффициенты, а $(x,y,z,w)$ - координаты возможных точек. Точка принпдлежит гиперплоскости, если выполняется равенство.

Требуется по данным точкам $$A(x_1,y_1,z_1,w_1), B(x_2,y_2,z_2,w_2), C(x_3,y_3,z_3,w_3), D(x_4,y_4,z_4,w_4)$$ составить уравнение гиперплоскости, т.е. найти нужные коэффициенты.

Уравнением плоскости будет являться решением данной матрицы: $$ \left|
		\begin{array}{cccc}
			x-x_1 & y-y_1 & z-z_1 & w-w_1     \\
			x_2-x_1 & y_2-y_1 & z_2-z_1 & w_2-w_1    \\
			x_3-x_1 & y_3-y_1 & z_3-z_1 & w_3-w_1      \\
			x_4-x_1 & y_4-y_1 & z_4-z_1 & w_4-w_1 
		\end{array}
	\right|=0

$$

Далее разберем другой способ. Пусть даны точка $A(x_1,y_1,z_1,w_1)$, и вектор $\vec N(x_2,y_2,z_2,w_2)$. Объявим гиперплоскость $P(ax + by + cz + dw + e=0)$.
Вектор $N$ будет являться нормалью к данной плоскости, т.е. будет справедливо следующее равенство: $$a=x_2, b=y_2, c=z_3, d=w_4.$$ Останется найти коэффициент $e$, который будет находиться подставлением координат данной точки $A$ в уравнение гиперплоскости. Т.е. $$e=-(x_2x_1+y_2y_1+z_2z_1+w_2w_1)$$
Таким образом, мы объявили гиперплоскость, используя введенные данные.
\subsection{Нахождение пересечения данной гипеплоскости с ребрами гиперкуба}
Для нахождения точек сечения надо найти пересечения данной гиперплоскости $P$ и всех ребер куба.
Допустим, имеем гиперкуб с координатами $$(\pm1, \pm1, \pm1, \pm1),$$ 16 вершин или 32 ребра. Для начала проверим, являются ли вершины гиперкуба точками сечения. Для этого берем последовательно все 16 вершин гиперкуба и подставляет их координаты в уравнение гиперплоскости.
\newline
$\begin{equation}
A(x_1,y_1,z_1,w_1) \\
P(ax + by + cz + dw + e = 0) \\
\left[
	\begin{array}{ccc}
		ax_1 + by_1 + cz_1 + dw_1 + e = 0 & \mbox{Вершина гиперкуба принадлежит сечению} 
		\\
		ax_1 + by_1 + cz_1 + dw_1 + e > 0 & \mbox{Вершина гиперкуба лежит выше плоскости сечения}
		\\
		ax_1 + by_1 + cz_1 + dw_1 + e < 0 & \mbox{Вершина гиперкуба лежит ниже плоскости сечния} 
	\end{array}
\end{equation}$
\\

Теперь, если данные точки, принадлежащие одному ребру лежат по разные стороны от сечения, то между ними есть точка, принадлежащая сечению. Если обе точки лежат по разные стороны от сечения, то значит данное ребро и плоскость сечения не имеют общих точек. Возможна также ситуация, когда вершина гиперкуба сама является точкой сечения.
\section{Описание работы программы}
\section{Примеры сечений}
\subsection{Тетраэдр}

\subsection{Пятигранная призма}
\subsection{Шестигранная призма}
0 1 0 1
1 0 1 1
0 1 1 1
0 0 0 0
\section{Тессеракт в культуре}
\subsection{В литература}
\subsection{В кинематографе}
\subsection{В мифологии(прим. жидорептилоиды)}
\section{Заключение}
\section{Список используемой литературы}
\section{Приложения}

\end{document}

